% Options for packages loaded elsewhere
\PassOptionsToPackage{unicode}{hyperref}
\PassOptionsToPackage{hyphens}{url}
\PassOptionsToPackage{dvipsnames,svgnames,x11names}{xcolor}
%
\documentclass[
  12pt,
]{article}
\usepackage{amsmath,amssymb}
\usepackage{lmodern}
\usepackage{iftex}
\ifPDFTeX
  \usepackage[T1]{fontenc}
  \usepackage[utf8]{inputenc}
  \usepackage{textcomp} % provide euro and other symbols
\else % if luatex or xetex
  \usepackage{unicode-math}
  \defaultfontfeatures{Scale=MatchLowercase}
  \defaultfontfeatures[\rmfamily]{Ligatures=TeX,Scale=1}
\fi
% Use upquote if available, for straight quotes in verbatim environments
\IfFileExists{upquote.sty}{\usepackage{upquote}}{}
\IfFileExists{microtype.sty}{% use microtype if available
  \usepackage[]{microtype}
  \UseMicrotypeSet[protrusion]{basicmath} % disable protrusion for tt fonts
}{}
\makeatletter
\@ifundefined{KOMAClassName}{% if non-KOMA class
  \IfFileExists{parskip.sty}{%
    \usepackage{parskip}
  }{% else
    \setlength{\parindent}{0pt}
    \setlength{\parskip}{6pt plus 2pt minus 1pt}}
}{% if KOMA class
  \KOMAoptions{parskip=half}}
\makeatother
\usepackage{xcolor}
\usepackage[margin=1in]{geometry}
\usepackage{graphicx}
\makeatletter
\def\maxwidth{\ifdim\Gin@nat@width>\linewidth\linewidth\else\Gin@nat@width\fi}
\def\maxheight{\ifdim\Gin@nat@height>\textheight\textheight\else\Gin@nat@height\fi}
\makeatother
% Scale images if necessary, so that they will not overflow the page
% margins by default, and it is still possible to overwrite the defaults
% using explicit options in \includegraphics[width, height, ...]{}
\setkeys{Gin}{width=\maxwidth,height=\maxheight,keepaspectratio}
% Set default figure placement to htbp
\makeatletter
\def\fps@figure{htbp}
\makeatother
\setlength{\emergencystretch}{3em} % prevent overfull lines
\providecommand{\tightlist}{%
  \setlength{\itemsep}{0pt}\setlength{\parskip}{0pt}}
\setcounter{secnumdepth}{5}
\usepackage{setspace}\doublespacing
\usepackage{booktabs}
\usepackage{longtable}
\usepackage{array}
\usepackage{multirow}
\usepackage{wrapfig}
\usepackage{float}
\usepackage{colortbl}
\usepackage{pdflscape}
\usepackage{tabu}
\usepackage{threeparttable}
\usepackage{threeparttablex}
\usepackage[normalem]{ulem}
\usepackage{makecell}
\usepackage{xcolor}
\ifLuaTeX
  \usepackage{selnolig}  % disable illegal ligatures
\fi
\IfFileExists{bookmark.sty}{\usepackage{bookmark}}{\usepackage{hyperref}}
\IfFileExists{xurl.sty}{\usepackage{xurl}}{} % add URL line breaks if available
\urlstyle{same} % disable monospaced font for URLs
\hypersetup{
  pdftitle={SIR 2025-2027 Specifications Economic Section},
  pdfauthor={Min-Yang Lee},
  colorlinks=true,
  linkcolor={Maroon},
  filecolor={Maroon},
  citecolor={Blue},
  urlcolor={blue},
  pdfcreator={LaTeX via pandoc}}

\title{SIR 2025-2027 Specifications Economic Section}
\author{Min-Yang Lee}
\date{August 13, 2024}

\begin{document}
\maketitle

\hypertarget{summary-and-housekeeping}{%
\section{Summary and Housekeeping}\label{summary-and-housekeeping}}

The 2025-2027 herring specification are needed to set the ABCs, ACLs,
and other fishery specifications. This is an application of the ABC
control rule that was put in place with Amendment 8. The previous specs
were set in Framework 8 and contain specifications through 2023.

A Supplemental Information Report (SIR) is being prepared. \emph{If}
there are no implementing regulations, then only a Regulatory
Flexibility Act Analysis is required. In this action, there is an
implementing regulation (New Brunswick Weir), therefore an EO 12866
section is required.

This analysis was written in RMarkdown. It can be found at
\url{https://github.com/NEFSC/READ-SSB-Lee-herring-analyses}

\setcounter{section}{8}
\setcounter{subsection}{11}

\setcounter{figure}{19}
\setcounter{table}{10}

\hypertarget{regulatory-impact-analysis-e.o.-12866}{%
\subsection{Regulatory Impact Analysis (E.O.
12866)}\label{regulatory-impact-analysis-e.o.-12866}}

The purpose of Executive Order 12866 (E.O. 12866, 58 FR 51735, October
4, 1993) is to enhance planning and coordination with respect to new and
existing regulations. This E.O. requires the Office of Management and
Budget (OMB) to review regulatory programs that are considered to be
``significant.'' E.O. 12866 requires a review of proposed regulations to
determine whether or not the expected effects would be significant,
where a significant action is any regulatory action that may:

\begin{itemize}
\tightlist
\item
  Have an annual effect on the economy of \$100 million or more, or
  adversely affect in a material way the economy, a sector of the
  economy, productivity, jobs, the environment, public health or safety,
  or State, local, or tribal governments or communities;
\item
  Create a serious inconsistency or otherwise interfere with an action
  taken or planned by another agency;
\item
  Materially alter the budgetary impact of entitlements, grants, user
  fees, or loan programs or the rights and obligations of recipients
  thereof; or
\item
  Raise novel legal or policy issues arising out of legal mandates, the
  President's priorities, or the principles set for the Executive Order.
\end{itemize}

In deciding whether and how to regulate, agencies should assess all
costs and benefits of available regulatory alternatives. Costs and
benefits shall be understood to include both quantifiable measures (to
the fullest extent that these can be usefully estimated) and qualitative
measures of costs and benefits that are difficult to quantify, but
nevertheless essential to consider.

The proposed action will set Annual Catch Limits and other fishery
specification for 2025-2027. In aggregate these changes will allow the
fishing industry to catch, land, and sell less herring. Lower revenues
are expected. Decreases in producer surplus accrue to the herring
fishing industry. Decreases in consumer surplus accrue to the users of
herring, these include the lobster industry. We do not project changes
in consumer or producer surplus. Changes in gross revenues from herring
are used as a proxy for these measures.

The rebuilding plan sets lower ACLS in order to rebuild the depleted
stock of herring to a biomass that can sustaim Maximum Sustainable
Yield. Furthermore, the stock of herring itself has value: it produces
future generations of fish and higher stock levels make harvesting less
costly. A bioeconomic model with that includes a stock-recruitment
relationship could be used to quantify the value of the changes in stock
levels. We do not undertake this.

\hypertarget{management-goals-and-objectives}{%
\subsubsection{Management Goals and
Objectives}\label{management-goals-and-objectives}}

\textcolor{red}{By reference}

\hypertarget{description-of-the-fishery-and-other-affected-entitites}{%
\subsubsection{Description of the Fishery and other affected
entitites}\label{description-of-the-fishery-and-other-affected-entitites}}

See Sections 3 and 4 for a description of the fishery

\hypertarget{statement-of-the-problem}{%
\subsubsection{Statement of the
Problem}\label{statement-of-the-problem}}

The New England Fishery Management Council adopted an ABC control rule
in Herring Amendment 8. The control rule prescribes the fishing
mortality rate (F) as a function of Spawning Stock Biomass. Framework 8
implemented the control rule for the 2021-2023 fishing years.

\hypertarget{economic-impacts-relative-to-the-baseline}{%
\subsubsection{Economic impacts relative to the
baseline}\label{economic-impacts-relative-to-the-baseline}}

The major change that would affect firms is a change in the Annual Catch
Limit from 23,961mt in 2025 to 2,710 mt, 6,854 mt, and 11,404mt in
2025-2027 respectively. Recent catches have been close to the ACLs, so
we assume that catch is equal to the ACLs in the future.

\hypertarget{prices-and-revenues}{%
\paragraph{Prices and Revenues}\label{prices-and-revenues}}

Framework 9 contained a simple econometric model that estimated a
relationship between (real 2019) prices and landings.

We have updated that model to include additional years of data and
normalize to Real 2023 prices. Wepply the results of that model of
prices to project future prices. Prices are in dollars per metric ton
and landings are expressed in thousands of metric tons. The first column
of Table \ref{regression_results} contains the model of prices that are
used to predict future prices\footnote{A least-squares regression will
  be produce biased estimates if prices and quantities are
  simultaneously determined. An Instrumental Variables estimator, where
  previous year's landings is used as an instrument for landings, can
  overcome this problem. A pair of log-transformed models are also
  estimated. The first column is the preferred specification and used
  for predictions. The other three columns are presented as robustness
  checks. The log-landings coefficient from the IV model is an
  elasticity and implies that an increase in landings of 1\% will reduce
  prices by 0.44\%.}. Based on the econometric model of prices,
predicted prices and revenues are calculated according to:

\begin{align}
\mbox{Predicted Price} &= 815 - 5.893*\mbox{landings}\label{eq:predicted_price}\\
\mbox{Predicted Revenue} &=  (815 - 5.893*\mbox{landings}) *\mbox{landings}\label{eq:predicted_landings}
\end{align}

The landings coefficient implies that, on average, an increase in
landings of 1,000 mt will reduce prices by \$5.89 per metric ton. The
data 2017 to 2020 was used to estimate this model.

\begin{table}[htbp]
  \begin{center}
    {
\def\sym#1{\ifmmode^{#1}\else\(^{#1}\)\fi}
\begin{tabular}{l*{4}{c}}
\hline\hline
                &       IV         &      OLS         &Log-Log IV         &  Log-Log         \\
\hline
landings (000 mt)&   -6.079\sym{***}&   -5.892\sym{***}&                  &                  \\
                &  (0.426)         &  (0.383)         &                  &                  \\
[1em]
log landings    &                  &                  &   -0.330\sym{***}&   -0.358\sym{***}\\
                &                  &                  & (0.0442)         & (0.0452)         \\
[1em]
Constant        &    893.9\sym{***}&    880.8\sym{***}&    7.399\sym{***}&    7.485\sym{***}\\
                &  (30.21)         &  (28.48)         &  (0.173)         &  (0.179)         \\
\hline
Observations    &       20         &       21         &       20         &       21         \\
\(R^{2}\)       &    0.919         &    0.926         &    0.777         &    0.767         \\
\hline\hline
\multicolumn{5}{l}{\footnotesize Standard errors in parentheses}\\
\multicolumn{5}{l}{\footnotesize \sym{*} \(p<0.05\), \sym{**} \(p<0.01\), \sym{***} \(p<0.001\)}\\
\end{tabular}
}

     \caption{Econometric model of herring prices used to project prices and revenues under baseline and propsoed conditions. \label{regression_results}}
  \end{center}
\end{table}

\hypertarget{baseline-description}{%
\subsubsection{Baseline Description}\label{baseline-description}}

The previous specifications included an ACL for 2025 was that is set at
23,961mt. This is one reasonable baseline against which to evaluate the
economic effects of the proposed action. Under the baseline, we project
prices of \textbf{\$674/metric ton} and gross revenues derived from
herring of \textbf{\$16.1M}.

\begin{table}

\caption{\label{tab:prices_revenues}Projected Landings (mt), Prices (Real 2023 USD/mt),  Revenues (Real 2023 USD/mt) and Revenue change relative to the baseline for  2025-2027 Specifications.}
\centering
\begin{tabular}[t]{lrrrl}
\toprule
\textbf{Year} & \textbf{Landings} & \textbf{Price} & \textbf{Revenue} & \textbf{Revenue Change}\\
\midrule
Baseline & 23,961 & \$674 & \$16,145,000 & NA\\
2025 & 2,710 & \$799 & \$2,165,000 & -\$13,979,000\\
2026 & 6,854 & \$775 & \$5,309,000 & -\$10,836,000\\
2027 & 11,404 & \$748 & \$8,528,000 & -\$7,617,000\\
\bottomrule
\end{tabular}
\end{table}

Table \ref{tab:prices_revenues} summarizes the annual prices, revenues,
and changes relative to the baseline.

The proposed action will result higher landings. The increase in
landings reduces herring prices slightly. Overall, we project an
\textbf{increase in revenues of -\$32,432,000} over the 2025-2027 time
period. When discounted at a 3\% discount rate, this corresponds to a
net present value \textbf{increase of -\$31,679,000}. Using a 7\%
discount rate, corresponds to a net present value \textbf{increase of
-\$30,759,000}.

\clearpage

\hypertarget{regulatory-flexibility-act-analysis}{%
\subsection{Regulatory Flexibility Act
Analysis}\label{regulatory-flexibility-act-analysis}}

\hypertarget{a-description-of-the-reasons-why-action-by-the-agency-is-being-considered.}{%
\subsubsection{A description of the reasons why action by the agency is
being
considered.}\label{a-description-of-the-reasons-why-action-by-the-agency-is-being-considered.}}

\textcolor{red}{By reference}

\hypertarget{a-succinct-statement-of-the-objectives-of-and-legal-basis-for-the-proposed-rule.}{%
\subsubsection{A succinct statement of the objectives of, and legal
basis for, the proposed
rule.}\label{a-succinct-statement-of-the-objectives-of-and-legal-basis-for-the-proposed-rule.}}

\textcolor{red}{By reference}

\hypertarget{number-of-small-entities}{%
\subsubsection{Number of Small
entities}\label{number-of-small-entities}}

The directly-regulated entities are the firms that currently hold at
least 1 Northeast US herring fishing permit (Categories A, B, C, D, or
E). Table \ref{tab:make_DRE_table} describes numbers of
directly-regulated entities, their main activities, and their revenues
from various sources. 739 small firms derive the majority of their
revenue from commercial fishing operations. 10 of the large firms derive
the majority of their revenue from commercial fishing activities.

There are 138 small firms that derive a majority of their revenue from
for-hire recreational fishing activities. The for-hire firms, while they
held at least one herring permit, did not derive any revenue from
herring.

\begin{table}[!h]

\caption{\label{tab:make_DRE_table}Number and Characterization of the  Directly Regulated Entities and Average Trailing Three Years of Revenue}
\centering
\begin{tabu} to \linewidth {>{\raggedright}X>{\raggedleft}X>{\raggedleft}X>{\raggedleft}X>{\raggedleft}X>{\raggedleft}X>{\raggedleft}X>{\raggedleft}X}
\toprule
\textbf{Size} & \textbf{Type} & \textbf{Firms} & \textbf{Vessels} & \textbf{Avg Gross Receipts} & \textbf{Avg Herring Receipts} & \textbf{25th pct Gross Receipts} & \textbf{75th pct Gross Receipts}\\
\midrule
Large & Fishing & 10 & 675 & \$19,094,000 & \$98,000 & \$15,574,000 & \$22,438,000\\
Small & Fishing & 739 & 5,870 & \$831,000 & \$7,000 & \$90,000 & \$1,030,000\\
Small & For-Hire & 138 & 890 & \$215,000 & \$0 & \$12,000 & \$159,000\\
\bottomrule
\end{tabu}
\end{table}

Table \ref{tab:make_DRE_table} suggests that there are many small firms
in the herring industry and that herring is minimally important to those
firms. While all of the small Fishing firms described in table
\ref{tab:make_DRE_table} hold a herring permit, many of these firms only
hold a category-D open access permit which has a 6,600lb possession
limit. These D-only firms have limited opportunity to increase their
catch of herring when catch limits increase. They are also less impacted
by closures of the fishery when the catch limits are reached, because
the possession limits are set to 2,000 pounds when this occurs. Many of
the firms described in Table \ref{tab:make_DRE_table} are not actively
engaged in the herring fishery. The herring fishery has had historically
low ACLs since 2018 and some firms have stopped participating in the
fishery. They may hold herring permits to preserve the option to fish.
The increases ACLs in the proposed action may or may not be high enough
to induce participation.

Table \ref{tab:Active_DREs} describes a subset of the directly-regulated
small entities, those that are both actively participating in the
herring fishery and hold a category A, B, C, or E herring permit.
Because there are fewer than 3 large firms, we only present a
description of the small firms. The small firms identified in table
\ref{tab:Active_DREs} are the firms most likely to be impacted by the
increases in ACLs in the proposed action.

\begin{table}

\caption{\label{tab:Active_DREs}Number and Characterization of the  Small, Active Directly Regulated Entities with A, B, C, or E permit, Trailing Three Years of Data.  Figures for the large firms cannot be presented to preserve confidentiality}
\centering
\begin{tabu} to \linewidth {>{\raggedright}X>{\raggedleft}X>{\raggedleft}X>{\raggedleft}X>{\raggedleft}X>{\raggedleft}X>{\raggedleft}X>{\raggedleft}X}
\toprule
\textbf{Size} & \textbf{Type} & \textbf{Firms} & \textbf{Vessels} & \textbf{Avg Gross Receipts} & \textbf{Avg Herring Receipts} & \textbf{25th pct Gross Receipts} & \textbf{75th pct Gross Receipts}\\
\midrule
Small & Fishing & 29 & 305 & \$1,510,000 & \$171,000 & \$495,000 & \$1,845,000\\
\bottomrule
\end{tabu}
\end{table}

\hypertarget{impacts-on-small-firms}{%
\subsubsection{Impacts on Small firms}\label{impacts-on-small-firms}}

\begin{table}

\caption{\label{tab:projected_revenues}Average projected and baseline gross reciepts and herring receipts for Small firms with A,B,C, or E permits  Figures for large firms cannot be show due to data confidentiality.}
\centering
\begin{tabu} to \linewidth {>{\raggedright}X>{\raggedleft}X>{\raggedleft}X>{\raggedleft}X>{\raggedleft}X>{\raggedleft}X>{\raggedleft}X>{\raggedleft}X}
\toprule
\textbf{Size} & \textbf{Type} & \textbf{Year} & \textbf{Firms} & \textbf{Projected Gross Receipts} & \textbf{Projected Herring Receipts} & \textbf{Baseline Gross Receipts} & \textbf{Baseline Herring Receipts}\\
\midrule
Small & Fishing & 2025 & 29 & \$2,442,000 & \$60,000 & \$2,828,000 & \$447,000\\
Small & Fishing & 2026 & 29 & \$2,529,000 & \$147,000 & \$2,828,000 & \$447,000\\
Small & Fishing & 2027 & 29 & \$2,618,000 & \$236,000 & \$2,828,000 & \$447,000\\
\bottomrule
\end{tabu}
\end{table}

To describe the effects of the changes in catch limits on small firms,
we project firm-level revenue corresponding to 2023-2026. We assume that
the share of herring landings for each firm is equal to their 2019-2021
average. We also assume the firms non-herring revenues are constant.
Inactive firms drop out and we focus on the vessels that have ABCE
permits. Table \ref{tab:projected_revenues} summarizes the projected
gross receipts, projected herring receipts, and baseline values. Figures
\ref{figure_boxR} and \ref{figure_boxH} illustrate the projected yearly
distribution of total and herring revenues from the Active vessels with
A,B,C, or E herring permits.

\begin{figure}
\centering
\includegraphics{SIR2025_Economic_Impacts_files/figure-latex/boxplotsR-1.pdf}
\caption{\label{figure_boxR}Projected Firm Level Revenue, Small firms
only}
\end{figure}

\begin{figure}
\centering
\includegraphics{SIR2025_Economic_Impacts_files/figure-latex/boxplots_H-1.pdf}
\caption{\label{figure_boxH}Projected Firm Level Herring Revenue, Small
firms only}
\end{figure}

\hypertarget{compliance-requirements}{%
\subsubsection{Compliance Requirements}\label{compliance-requirements}}

\textcolor{red}{A description of the projected reporting, record-keeping, and other compliance requirements of the proposed rule, including an estimate of the classes of small entities which will be subject to the requirements of the report or record.}

\hypertarget{duplications}{%
\subsubsection{Duplications}\label{duplications}}

\textcolor{red}{An identification, to the extent practicable, of all relevant Federal rules, which may duplicate, overlap, or conflict with the proposed rule.}

\newpage

\end{document}
